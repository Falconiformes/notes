\documentclass{beamer}
\usetheme[shownavsym]{AAUsimple}
\usefonttheme{structurebold}
%%%%%%%%%%%%%%%%%%%%%%%%%%%%%%%%%%
\usepackage[utf8]{inputenc}
\usepackage[dutch]{babel}
\usepackage[T1]{fontenc}
\usepackage{tikz}
\usetikzlibrary{positioning}
\usepackage{hyperref}
\usepackage{lmodern}
\usepackage{microtype}
\usepackage{siunitx}
%%%%%%%%%%%%%%%%%%%%%%%%%%%%%%%%%%
% Gekleurde hyperlinks
\newcommand{\chref}[2]{%
	\href{#1}{{\usebeamercolor[bg]{AAUsidebar}#2}}%
}
\title[Fysiologie]{Cardiovasculair homeostase}

\date{\today}

\author[Edon Namani, et. al]
{%
	Edon Namani\\
	\href{mailto:e.namani@student.rug.nl}{{\tt e.namani@student.rug.nl}}
}

\institute[%
	Faculteit Gezondheidswetenschappen\\
	Rijksuniversiteit Groningen\\
	Nederland
]
{%
	Faculteit Gezondheidswetenschappen\\
	Rijksuniversiteit Groningen\\
	Nederland

}

\pgfdeclareimage[height=1.5cm]{titlepagelogo}{AAUgraphics/semper_invicta.pdf}
\titlegraphic{%
	\pgfuseimage{titlepagelogo}
}

\begin{document}
% Titel
{\aauwavesbg%
	\begin{frame}[plain,noframenumbering]
	\titlepage
\end{frame}}
%%%%%%%%%%%%%%%

%TOC
\begin{frame}{Agenda}{}
	\tableofcontents
\end{frame}
%%%%%%%%%%%%%%%
\section{Definitie homeostase}
\begin{frame}{Definitie homeostase}
    \begin{block}{Definitie}
        Het lichaam houdt waardes van variabelen binnen een bepaald interval, bv. $[4.4, 6.1] \si{\milli\mol}$ (waarde-interval) bloedsuikerspiegel (variabel). De waarde van de variabel wordt gedetecteerd door een sensor. De hersenen checken of de waarde binnen het waarde-interval zit. Bij afwijkingen sturen de hersenen acties aan die de waarde weer doen herstellen.

        De variabelen in het cardiovasculair systeem:
        \begin{enumerate}
            \item Arteriële druk, $P_a$
            \item Veneuze overschot $\rightarrow \int (Q_{\text{terugkeer}} - Q_{\text{hart}}) dt$
            \end{enumerate}
    \end{block}
\end{frame}
\section{Verschil tussen arteriën en venen}
\begin{frame}{Verschil tussen arteriën en venen}
    \begin{columns}
        \column{.5\textwidth}
        \begin{block}{Arteriën}
            \begin{itemize}
                \item Functie $\rightarrow$ Verdelen van bloed
                \item Lage compliantie
                \item Hoge weerstand
            \end{itemize}
        \end{block}
    \column{.5\textwidth}
    \begin{block}{Venen}
        \begin{itemize}
            \item Functie $\rightarrow$ Verzamelen van bloed
            \item Hoge compliantie $\rightarrow$ groot bloedvolume/reservoir
            \item Lage weerstand
        \end{itemize}
    \end{block}
    \end{columns}
\end{frame}
\section{Cardiovasculair homeostase lus}
\begin{frame}{Cardiovasculair homeostase lus}{Soort van...}
    \begin{tikzpicture}[every node/.style={align=center, minimum size=1cm},bluestyle/.style={text=white,fill=blue!50,draw=blue}]
        \matrix[ampersand replacement=\&, column sep=.66cm, row sep=1cm] {
            \& \& \& \node[bluestyle] (hypoxia) {Hypoxia}; \& \\ 
            \node[bluestyle] (Vcompliantie) {$C_v$ | $V_{tot}$}; \&
                \node[text=white,fill=red!50,draw=red] (Voverschot) {Veneus\\ overschot}; \&
            \node[bluestyle] (CO) {$Q_{\text{hart}}$};
            \& \node[bluestyle] (Vaterie) {Arteriële\\ overschot};
            \& \node[bluestyle] (Resistentie) {$R_{perifeer}$};\\
            % row 2
                \& \node[bluestyle] (Pra) {$P_{ra}$}; 
            \& \node[bluestyle] (Vslag) {$SV$}; 
            \& \node[bluestyle] (freq) {$f_{\text{hart}}$};
            \& \node[text=white,fill=red!50,draw=red] (AP) {$P_a$};\\
        };
    \begin{scope}[->,thick,>=stealth]
        \draw[-|,red] (CO) -- (Voverschot);
        \draw (CO) -- (Vaterie);
        \draw (Voverschot) -- (Pra);
        \draw (Vslag) -- (CO);
        \draw (Pra) -- (Vslag);
        \draw (Pra) -| (Vcompliantie) node[anchor=north,midway,text=beamer@normaltextcolor] {ANP};
        \draw[-|,red] (Vcompliantie) -- (Voverschot);
        \draw (freq.north west) -- (CO.south east);
        \draw[-|,red] (AP) -- (freq);
        \draw[-|,red] (AP) -- (Resistentie);
        \draw (Resistentie) -- (Vaterie);
        \draw (Vaterie.south east) -- (AP.north west);
        \draw[-|,red] (hypoxia) -| (Resistentie);
        \draw[-|,red] (Vaterie) -- (hypoxia);
        \end{scope}
    \end{tikzpicture}
\end{frame}
\section{Debiet-curves}
\begin{frame}{Debiet-curves}{Veneuze terugkeer \& Hart debiet}
    \only<1>{
    \begin{columns}
        \column{.45\textwidth}
        \begin{tikzpicture}[>=stealth,thick]
            \draw[<->] (0,5) node[anchor=south] {$Q_{terugkeer} $ (\si{\litre\per\minute})} -- ++(0,-5) -- (5,0) node[anchor=west] {$P_{ra}$ (mm Hg)};
            \foreach \y/\t/\d in {0/0/-4,1/4/0,2/8/4,3/12/8,4/16/12}{
                \draw (1mm,\y) -- (-1mm,\y) node[anchor=east] {\t};
            \draw (\y,1mm) -- (\y,-1mm);
                \node[anchor=north] at (\y,0) {\d};
        }
        \draw[rounded corners,beamer@headercolor] (0,1.4) -- ++(.8,0) -- ++(2.0,-1.4) node[fill=red,circle,inner sep=0.9] {};
        \end{tikzpicture}
        \column{.5\textwidth}
        \begin{block}{Analyse grafiek}
            \begin{itemize}
            \item De rechte $\rightarrow \frac{1}{R_{perifeer}}$ 
            \item Snijpunt met $P$-as geeft de groote van $C_v$ en $V_t$ aan.
            \item Plateau $\rightarrow P_{ra} < P_{thoraxholte}$
            \end{itemize}
        \end{block}

    \end{columns}
}
\only<2>{
    \begin{columns}
        \column{.45\textwidth}
        \begin{tikzpicture}[>=stealth,thick]
            \draw[<->] (0,5) node[anchor=south] {$Q_{hart} $ (\si{\litre\per\minute})} -- ++(0,-5) -- (5,0) node[anchor=west] {$P_{ra}$ (mm Hg)};
            \foreach \y/\t/\d in {0/0/-4,1/4/0,2/8/4,3/12/8,4/16/12}{
                \draw (1mm,\y) -- (-1mm,\y) node[anchor=east] {\t};
            \draw (\y,1mm) -- (\y,-1mm);
                \node[anchor=north] at (\y,0) {\d};
        }
        \draw[beamer@headercolor] (0,0) .. controls +(0:2) and +(180:5).. (4,4);
        \end{tikzpicture}
        \column{.5\textwidth}
        \begin{block}{Analyse grafiek}
            \begin{itemize}
                \item Eind-diastolisch volume afhankelijk van $P_{transmuraal}$
                \item Verandering maximum door (immiterende) sympathismus
            \end{itemize}
        \end{block}
    \end{columns}
}
\only<3>{%
    \begin{columns}
        \column{.45\textwidth}
        \begin{tikzpicture}[>=stealth,thick]
            \draw[<->] (0,5) node[anchor=south] {$Q$ (\si{\litre\per\minute})} -- ++(0,-5) -- (5,0) node[anchor=west] {$P_{ra}$ (mm Hg)};
            \foreach \y/\t/\d in {0/0/-4,1/4/0,2/8/4,3/12/8,4/16/12}{
                \draw (1mm,\y) -- (-1mm,\y) node[anchor=east] {\t};
            \draw (\y,1mm) -- (\y,-1mm);
                \node[anchor=north] at (\y,0) {\d};
        }
        \draw[rounded corners,green] (0,1.4) -- ++(.8,0) -- ++(2.0,-1.4);
        \draw[beamer@headercolor] (0,0) .. controls +(0:2) and +(180:5).. (4,4) node[pos=.393,circle,fill=red,inner sep=1.2] {};
        \draw[blue,dashed] (0,0) .. controls +(0:1) and +(180:4).. (4,2) node[pos=.576,circle,fill=cyan,inner sep=1.2] (pos2) {};
        \path (0,0) .. controls +(0:1) and +(180:4).. (4,2) node[pos=.470,circle,fill=cyan,inner sep=1.2] (pos1) {};
        \draw (pos1) -| (pos2);
        \end{tikzpicture}
        \column{.5\textwidth}
        \begin{block}{Analyse combi-grafiek}
            \begin{itemize}
                \item Op lange termijn $Q_{terugkeer} = Q_{hart}$
            \item Bij hartfaal tijdelijk $Q_{terugkeer} > Q_{hart}$
            \item $Q_{terugkeer}$ is immers afhankelijk van $Q_{hart}$
            \end{itemize}
        \end{block}
    \end{columns}
}
\end{frame}
{\aauwavesbg
\begin{frame}[plain,noframenumbering]
	\finalpage{ --- Gjithmon pafitu ---}
\end{frame}}
\end{document}
