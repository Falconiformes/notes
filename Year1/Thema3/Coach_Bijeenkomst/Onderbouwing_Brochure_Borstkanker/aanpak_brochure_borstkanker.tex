\documentclass{scrartcl}
\usepackage{lmodern}
\usepackage[dutch]{babel}
\usepackage{microtype}
\usepackage{dtk-logos}
\usepackage{biblatex}
\usepackage{csquotes}
\usepackage{hyperref}
\KOMAoptions{DIV=calc}
%%
\addbibresource{./bib.bib}
\nocite{*}
%%
\title{Aanpak schrijven borstkankerbrochure}
\author{Edon Namani (S3611833)\thanks{Een arts is een idealist \'en een realist.}}
\date{\today}
\subject{LC DZ Taak 1.3}
%%
\begin{document}
\maketitle

Voor het schrijven van een brochure over borstkanker waren er drie obstakels die overwonnen moesten worden. Deze waren het ontwerp van een aantrekkelijk, typografisch kwalitatieve folder, vergaring van literatuur en de vergaarde informatie op de epistemologische graad van de pati\"ent vertonen.

\section{Formaat van brochure}
Criteria betreffende het formaat van brochure waren niet verschaft. Door deze vrijheid heb ik besloten om de brochure in \XeLaTeX\ te maken. Dit heeft mogelijk gemaakt om een brochure van hoog typografische standaard die bestaat uit 6 subpagina's en een kleurenpalette corresponderend met borstkanker te vervaardigen.

\section{Vergaring van literatuur}
In de wetenschappelijke literatuur is de pyramide van bewijs bekend. De pyramide stelt dat overzichtsartikelen de grootste waarde als bewijsstuk hebben. De overzichtsartikelen verworf ik op \href{https://www.cochranelibrary.com/}{cochrane website} en \href{https://www.ncbi.nlm.nih.gov/pubmed/}{pubmed website} met een filter op overzichtsartikel en de juiste MeSH termen. Effectiviteit van een bevolkingsonderzoek voor borstkanker, verlaging van mortaliteit te wijten aan borstkanker, en de nadelen van een bevolkingsonderzoek voor borstkanker, overdiagnose en fout-positieven, zijn afkomstig van deze overzichtsartikelen. Statistiek van bevolkingsmammografie zijn afkomstig van website van \textsc{RIVM}. Ten slotte zijn gegevens van incidentie en prevalentie afkomstig van kwf-organisatie.

\section{Vertaling naar voor leken vatbare informatie}
Zoals geuit in het fameuze boek \textit{Nineteen Eighty-Four} wordt de omvang van het denken en begrijpen door de graad van meestering van vocabularium bepaald. De 50-jarige vrouwen hebben over het algemeen een lagere graad van meestering van vocabularium dan meeste pre-artsen. Derhalve vermeed ik gebruik van infrequente en esoterische woorden. Zo is incidentie vertaald naar ``het krijgen van borstkanker'' en prevalentie naar ``het hebben van borstkanker''. Een ander voorbeeld is ``onterecht verklaard borstkanker te hebben, terwijl een borstkankergezwel afwezig is'' voor fout-positief. Verder trachtte ik de zinnen zover als mogelijk is in de actieve vorm te formuleren en vorming van bijzinnen te vermijden.

In ene geval vertelt een woord meer dan 1000 plaatjes, in een ander geval juist een plaatje meer dan 1000 woorden. Prevalentie, statistische waarden van bevolkingsmammografie en overdiagnose kunnen vatbaar gemaakt worden door ze te visualiseren in een hi\"erarchische boomstructuur. Hierbij reflecteert de grootte van een groepseenheid op een bepaalde hi\"erarchische niveau de kans dat een 50-jarige vrouw deel van deze groepseenheid uitmaakt. Verder waren de groepseenheden die nadeel van het bevolkingsonderzoek ondervinden toepasselijk rood gekleurd.

In mijn ogen is het belang van pati\"entgerichte communicatie bij het informeren van patiënten over screening op kanker dat de pati\"ent zelf de mogelijkheid heeft om de knoop door te hakken. Dat wil zeggen dat de pati\"ent op een beschouwende en vatbare wijze de kans op schade of winst bij participatie aan een bevolkingsonderzoek ge\"{i}nformeerd krijgt.
\printbibliography
\end{document}
