\documentclass[foldmarks=true,foldmarks=blmtP,fromphone,
fromemail,fromlogo,version=last,sections]{scrlttr2}
\usepackage{array,multirow, booktabs}%%Tabellenmaker 1.Array voor horizontale lijn en arraystretch, 2. Multirow en Multicolumn voor merging rijen en columnen respectievelijk, 3.Booktabs voor code commands in tabel.
\usepackage{libertine}
\usepackage{graphicx}
\usepackage[dutch]{babel}
\usepackage{microtype}
\usepackage[pdftex,
            pdfauthor={Edon Namani, Jolien Oud},
            pdftitle={Re-integratie van mr.~Visser},
            pdfsubject={Re-integratie plannen},
            pdfkeywords={Re-integratie,beroepsziekte,arbeidsongeschiktheid},
            pdfproducer={Latex with hyperref},
            pdfcreator={pdftex}]{hyperref}
\usepackage{graphicx}
\begin{document}
\setkomavar{fromname}{Edon Namani --- Jolien Oud
                     }
\setkomavar{fromaddress}{Churcilllaan 11\\
			Utrecht 3500 GC%
}
\setkomavar{fromphone}{+31\,6\,28~21\,49\,68}
\setkomavar{fromemail}{e.namani@student.rug.nl --- j.oud.3@student.rug.nl}
\setkomavar{fromlogo}{\includegraphics[scale=.1]{logo_bedrijfsarts.pdf}}
\setkomavar{location}{\raggedright%
	Bedrijfsartsen des syndicaat \textsc{NVAB}\\
	sedert 31.07.2016\\
}
\setkomavar{subject}{Re-integratieplan van dhr.~Visser}
\begin{letter}{%
	Ristorante italiano\\
	Dummyadres 9\\
	Groningen\\
	0000 ZZ%
}
\opening{Geachte heer Venema,}
Een van uw werknemers kan zijn werkzaamheden niet voortzetten door een medisch probleem. In deze adviesbrief bespreken wij de bijdrage van het werk aan het medisch probleem en de mogelijke re-integratie plannen.
\section{Diagnose van het medisch probleem}
Bij de werknemer is de diagnose chronisch allergisch en effectief conctacteczeem gesteld. Contacteczeem is een ontsteking van de huid door het contact met irriterende en/of allergische stoffen. In het geval van de werknemer zijn deze allergische stoffen plantaardige producten, denk hierbij aan bijvoorbeeld groentesappen. Contacteczeem veroorzaakt roodheid, zwelling, blaren, wondjes, schilfering en eeltvorming. 

\section{Bijdrage van werk aan contacteczeem}
Werkers in de horeca lijden vaker aan contacteczeem dan werkers in andere beroepen. Dit laat zien dat het contacteczeem veroorzaakt kan worden door het werk in de horeca. Om te bepalen of het eczeem daadwerkelijk veroorzaakt wordt door het werk wordt er gekeken naar 2 dingen:
\begin{enumerate}
    \item Het eczeem verergert bij het uitvoeren van werkzaamheden.
    \item Klachten van het eczeem verminderen wanneer de werknemer vrij is.
\end{enumerate}
Verder geeft de werknemer aan dat zijn onderkok lijdt aan vergelijkbare klachten.

De ingrediënten van de gerechten die de werknemer maakt, bevatten stoffen waarvoor de werknemer allergisch is. De werknemer geeft aan dat hij 5 achtereenvolgende dagen 6 uur lang met deze ingrediënten in contact is. Omdat de werknemer geen handschoenen draagt is het contact met de ingrediënten direct op de huid. Ook maakt de werknemer op zijn werkdagen 1 uur schoon, de zeep die hij daarvoor gebruikt werkt irriterend en verergert het contacteczeem. Wanneer de werknemer vrij is komt hij niet in contact met de allergische stoffen. 

Verder gebruikt de werknemer geen medicijnen die de kans op contacteczeem vergroten en heeft hij hiervoor nog nooit eerder eczeem gehad. 

\emph{Uit al het bovenstaande blijkt dus dat het contacteczeem veroorzaakt wordt door werkzaamheden.}

\section{Advies werkmogelijkheden}
Ons advies is dat de werknemer zijn huidige functie niet moet uitvoeren. De genezing van het contacteczeem zal dan sneller verlopen. De genezing van het contacteczeem duurt gemiddeld een half jaar. Zolang de genezing duurt, kan de werknemer andere werkzaamheden uitvoeren binnen de firma, als dit niet lukt kan het ook bij een andere firma. Na de genezing kan de werknemer zijn huidige functie weer uitvoeren, maar moet contact met de irriterende stoffen wel vermeden worden. Daarom zou op de werkplaats vaseline beschikbaar moeten zijn en vinyle handschoenen gevoerd met katoen. Deze maatregelen zorgen ervoor dat het contacteczeem niet weer terugkomt. 

De werknemer kan zijn eczeem behandelen door cortisonezalf en eventueel het ondergaan van fototherapie, wat verder besproken zal worden met de huisarts/dermatoloog. Deze maatregelen kunnen ervoor zorgen dat de werknemer weer volledig arbeidsgeschikt wordt. 
\section{Kosten \& uitkering}
De kosten voor u omvat het volgende:
\begin{enumerate}
    \item Volgens de wet van doorloonbetaling moet gedurende re-integratie van de werknemer het loon voor minstens 2 jaar 70 \% doorbetaald worden.
    \item De vaseline en de handschoenen
    \item De kosten voor het schrijven van dit adviesplan.
\end{enumerate}

    De kosten van de diagnose van chronisch allergisch en irriterend contacteczeem, de cortisonezalf en lichttherapie zijn voor de werknemer. Deze kunnen bij de verzekeraar gedeclareerd worden.

    Ten slotte komt de werknemer na de re-integratie door zijn arbeidsgeschiktheid niet in aanmerking voor \textsc{uwv}-uitkering, zoals uitgestippeld staat in Wet werk en inkomen naar arbeidsvermogen (\textsc{wia}).

\closing{Hopende voldoende raad te hebben verschaft}
\setkomavar*{enclseparator}{Disclaimer}
\encl{Met instemming van de individu in kwestie is vertrouwelijke informatie in deze correspondentie gepubliceerd.}
\cc{Dhr.~Visser}
\vspace{\baselineskip}
\centering
    \begin{tabular}{*4l}
        \toprule
            \multicolumn{2}{l}{\textbf{Organisatie gegevens}} & \multicolumn{2}{l}{\textbf{Werknemer gegevens}}\\
        \midrule
            Organisatie      & Ristorante italiano               & Naam          & Dhr.~J~Visser        \\
            Adres            & Dummyadres 9                      & Geboortedatum & 11-01-1989           \\
            Postcode, plaats & 0000 ZZ, Groningen                 & Cliëntcode    & XXXXXXX              \\
            Contactpersoon   & Dhr.~Bossy                        & Adres         & Dummystraat 8 Morra  \\
            E-mail adres     & \texttt{\small rist.pers@xyz.com} & Functie       & Chef kok             \\
        \bottomrule
    \end{tabular}
\vfil
    \emph{LC DZ Taak 1.5\\Casus 3: 30-jarige kok\\Edon Namani \& Jolien Oud}
\vfil
\end{letter}
\end{document}
