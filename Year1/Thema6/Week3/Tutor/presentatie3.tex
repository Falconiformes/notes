\documentclass{beamer}
\usetheme[shownavsym]{AAUsimple}
\usefonttheme{structurebold}
%\definecolor{BLUE}{RGB}{0,0,110}
%\definecolor{light_BLUE}{RGB}{118,118,134}
%\definecolor{CYAN}{RGB}{68,65,77}
%\definecolor{UBCblue}{rgb}{0.04706, 0.13725, 0.26667} % UBC Blue (primary)
%\definecolor{UBCgrey}{rgb}{0.3686, 0.5255, 0.6235} % UBC Grey (secondary)
%
%%\setbeamercolor{palette primary}{bg=UBCblue,fg=white}
%%\setbeamercolor{palette secondary}{bg=UBCblue,fg=white}
%%\setbeamercolor{palette tertiary}{bg=UBCblue,fg=white}
%%\setbeamercolor{palette quaternary}{bg=UBCblue,fg=white}
%%\setbeamercolor{structure}{fg=UBCblue} % itemize, enumerate, etc
%%\setbeamercolor{section in toc}{fg=UBCblue} % TOC sections
%\setbeamercolor{structure}{fg=UBCblue}
%\setbeamercolor{frametitle}{bg=UBCblue,fg=white}
%\setbeamercolor{AAUsimple}{fg=UBCgrey,bg=UBCblue}
%\setbeamercolor{normal text}{fg=CYAN}
%%%%%%%%%%%%%%%%%%%%%%%%%%%%%%%%%%
\usepackage[utf8]{inputenc}
\usepackage[dutch]{babel}
\usepackage[T1]{fontenc}
\usepackage{tikz}
\usepackage{hyperref}
\usepackage{lmodern}
\usepackage{microtype}
\tikzset{>=stealth}
\usepackage{circuitikz}
%%%%%%%%%%%%%%%%%%%%%%%%%%%%%%%%%%
% Gekleurde hyperlinks
\newcommand{\chref}[2]{%
	\href{#1}{{\usebeamercolor[bg]{AAUsidebar}#2}}%
}
\title[Cardiologie]{Vloeistofdynamica in fysiologie}
\date{\today} 
\author[Edon Namani, et. al]
{%
	Edon Namani\\
	\href{mailto:e.namani@student.rug.nl}{{\tt e.namani@student.rug.nl}}
}

\institute[%
	Faculteit Gezondheidswetenschappen\\
	Rijksuniversiteit Groningen\\
	Nederland
]
{%
	Faculteit Gezondheidswetenschappen\\
	Rijksuniversiteit Groningen\\
	Nederland

}

\pgfdeclareimage[height=1.5cm]{titlepagelogo}{AAUgraphics/semper_invicta.pdf}
\titlegraphic{%
	\pgfuseimage{titlepagelogo}
}

\begin{document}
% Titel
{\aauwavesbg%
	\begin{frame}[plain,noframenumbering]
	\titlepage
\end{frame}}
%%%%%%%%%%%%%%%

%TOC
\begin{frame}{Agenda}{}
	\tableofcontents
\end{frame}
%%%%%%%%%%%%%%%
\section{Debiet}
\begin{frame}{Debiet}
    \begin{columns}
        \column{.3\textwidth}
        \begin{tikzpicture}[thick]
            \usebeamercolor{frametitle}
            \draw[dashed] (0,-.5) arc (-90:90:.25 and .5);
            \draw (0,-.5) arc (270:90:.25 and .5);
    \draw (0,-.5) -- (2.5,-.5);
    \draw (0,.5) -- (2.5,.5);
        \draw (2.5,0) ellipse (.25 and .5);
        \draw[fill,draw=bg] (.5,.-.5) arc (-90:-270:.25 and .5) -- (1.5,.5) -- +(0,-1) -- (.5,-.5); 
        \filldraw[bg] (1.5,0) ellipse (.25 and .5) node[fg] {$A$};
        \draw[<->] (.5,-.75) -- +(1,0) node[midway, anchor=north] {$h$};
        \draw[->] (1.75,0) -- +(.5,0) node[anchor=west] {$\vec{v}$};
    \end{tikzpicture} 
    \column{.7\textwidth} 
    \only<1>{%
    \begin{block}{Definitie}
        Debiet ($Q$) is hoeveelheid bloed die per tijdseenheid een bepaald punt passeert. Mathematisch gedefinieerd:
        \begin{equation*}
            Q = \frac{dV}{dt} 
        \end{equation*}
        Hierin is:
        \begin{itemize}
            \item $V$ het volume
            \item $t$ de tijd
        \end{itemize}
    \end{block}
}
\only<2>{%
    \begin{block}{Relatie stroomsnelheid en debiet}
         \begin{equation*}
             Q = \frac{dV}{dt}=\frac{Adh}{dt}=Av 
        \end{equation*}  
        Voor elk categorie bloedvat, bv. capillairen, is $Q$ gelijk. Maar de dwarsdoorsnede oppervlakte is verschillend. Capillairen hebben de grootste $A$ en dus ook de laagste $v$.
    \end{block}
}
\end{columns} 
\end{frame}
\section{Weerstand en Debiet}
\begin{frame}{Weerstand \& Debiet}
     Debiet is proportioneel met het drukverschil tussen twee verschillende punten en invers proportioneel met de weerstand tussen deze twee punten:
     \begin{equation*}
         Q = \frac{\Delta P}{R} \rightarrow \frac{\pi\Delta P r^4}{8\eta l}
     \end{equation*}
     Bloed stroomt door parallelle en in serie staande weerstanden/bloedvaten. 

     \alert{$R$ voor serie}:
     \begin{equation*}
         R_{\text{tot}} = \sum_{i=1}^{n} R
     \end{equation*}
     \alert{$R$ voor parallel}:
     \begin{equation*}
         \frac{1}{R_{\text{tot}}} = \sum_{i=1}^{n} \frac{1}{R}
     \end{equation*}
 \end{frame}
{\aauwavesbg
\begin{frame}[plain,noframenumbering]
    \finalpage{C'est finit}
\end{frame}}
\end{document}
