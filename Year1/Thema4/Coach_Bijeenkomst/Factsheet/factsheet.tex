\documentclass[twocolumn]{scrartcl}
\usepackage{libertine}
\usepackage[dutch]{babel}
\usepackage{microtype}
%\usepackage[pdfauthor={Edon Namani},
%            pdftitle={Factsheet antibiotica resistentie},
%            pdfsubject={Antibiotica resistentie},
%            pdfkeywords={Antibiotica,antibiotica resistentie,factsheet},
%            pdfproducer={Latex with hyperref},
%            pdfcreator={pdflatex}]{hyperref}
\usepackage{graphicx}
\usepackage{biblatex,csquotes}
\bibliography{bib.bib}
\title{Antibiotica resistentie factsheet}
\author{Edon Namani (S3611833)\thanks{Een arts is een idealist én een realist.}}
\subject{LC DZ Taak 1.4}
\date{\today}
\begin{document}
\maketitle
\section*{Definitie antibiotica resistentie}
Antibiotica resistentie duidt op het verlies van het dodend effect van een bepaalde antibioticum op een bepaalde bacteriesoort. Inherente resistentie van zekere bacteriesoort wordt dus in deze definitie uitgesloten.
\section*{Omvang van het probleem}
Door antibiotica resistentie zullen in 2050 10~miljoen mensen per jaar sterven. Resistentie zorgt ervoor dat veel mensen sterven aan post-operatieve infecties en immuungecomprimenteerde patiënten, hetzij erfelijk hetzij medisch geïnduceerd, worden niet meer beschermd tegen bacteriën.
\section*{Oorzaken van antibiotica resistentie}
Gebruik van antibiotica fungeert als een selectiedruk op de verschillende stammen in een bacteriesoort en op de verschillende bacteriesoorten. Overvloedig gebruik van antibiotica leidt tot extinctie van alle resistente bacteriën en overleving van alle resistente bacteriën. Dit overvloedig gebruik wordt naast gebrek van rationale door de volgende factoren beïnvloed:

\begin{itemize}
    \item Medicalisering is een proces waarbij veelvuldig normale condities als pathologisch verklaard worden. Een subliem voorbeeld is ontwikkeling van acne in de pubertijd. Zij wordt gezien als een ziekte en heeft derhalve een behandeling. Eén van deze behandeling is antibioticum. Hierdoor leidt het ``helende'' handelen van de arts tot meer antibiotica resistentie en uiteindelijk tot meer zieken en doden. Dit wordt iatrogenese genoemd;
    \item Farmaceuticalisering is ook een proces waarbij elke kwaal een farmaceutische middel ter behandeling vereist ongeacht zijn effectiviteit.
\end{itemize}

\subsection*{Verkrijgen van antibiotica resistente genen}
Meeste bacteriën krijgen hun resistente eigenschap niet door gain-of-function mutaties maar door ontvangst van antibiotica resistente vectoren. Deze vectoren kunnen op drie wijzen ontvangen worden:

\begin{itemize}
    \item Transformatie $\rightarrow$ Opname van extracellulaire genetisch materiaal;
    \item Transductie $\rightarrow$ Bacteriofaag incorporatie in bacterieel genoom;
    \item Conjugatie $\rightarrow$ Uitwisseling van plasmiden tussen bacteriën.
\end{itemize}

\section*{Oplossing tegen antibiotica}
\subsection*{Disease \& illness}
De concepten disease en illness zijn vergelijkbaar met de concepten subjectiviteit en objectiviteit. Het hebben van een disease duidt op een abnormale moleculaire/cellulaire order. Illness daarentegen duidt op de ervaring van de patiënt van de disease. Het gevoel van ziek zijn, hangt uiteindelijk af van de geest. Illness kan in sommige gevallen precies het tegenovergestelde van psychosomatisme zijn. Sommige patiënten met DM II kunnen pijnloos door het leven gaan, terwijl bij een psychosomatische aandoening een gebrek van fysiologische afwijking toch een gevoel kan geven van ziek zijn.

Deze twee concepten werden in het onderzoek van Cals toegepast. Het aandeel van correcte voorschrijving van antibiotica tegen lageluchtweginfecties door huisartsen werd onderzocht door CRP-test en verbeterde communicatie vaardigheden. De combinatie van CRP-test(disease) en verbeterde communicatie vaardigheden(illness) resulteerden in de meeste correcte voorschrijvingen.

De complementaire relatie van disease en illness leiden tot een rationelere inzet van antibiotica. Zo is de CRP-waarde een continu variabel, waarbij de afkapwaarde min of meer arbitrair gekozen zijn. Door deze eigenaardigheid leidt, als enige leidraad genomen, veelvuldig tot een vorm van medicalisering, overbehandeling, zoals voorschrijven van antibiotica bij een viraal veroorzaakte acute bronchitis. Het betrekken van psychosomatisch aspect van een ziekte kan een dergelijke medicalisering voorkomen.
\nocite{*}
\printbibliography
\end{document}
