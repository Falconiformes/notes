\documentclass[foldmarks=true,foldmarks=blmtP,fromphone,
fromemail,fromlogo,version=last]{scrlttr2}
\usepackage{libertine}
\usepackage[dutch]{babel}
\usepackage{microtype}
\usepackage[pdftex,
            pdfauthor={Edon Namani},
            pdftitle={Bijscholing huisarten antibiotica voorschrijven},
            pdfsubject={Correct voorschrijven antibiotica},
            pdfkeywords={Antibiotica,antibiotica resistentie,bijscholing},
            pdfproducer={Latex with hyperref},
            pdfcreator={pdflatex}]{hyperref}
\usepackage{graphicx}
\begin{document}
\setkomavar{fromname}{Edon Namani}
\setkomavar{fromaddress}{Antonius Deusinglaan 2\\
			Groningen\\
			9713 AW%
}
\setkomavar{fromphone}{+31\,6\,28~21\,49\,68}
\setkomavar{fromemail}{e.namani@student.rug.nl}
\setkomavar{fromlogo}{\includegraphics{chi_rho.pdf}}
\setkomavar{location}{\raggedright%
	Lid des syndicat \textsc{unkwn}\\
	sedert 31.07.2016\\
	praeses 2017-2018%
}
\begin{letter}{%
	Nederlands Huisartsen Genootschap\\
	Mercatorlaan 1200\\
	Utrecht\\
	3528 BL%
}
\opening{Geachte praeses,}
in het nog niet zoverre verleden was de angst om te sterven aan een bacteri\"{e}le infectieziekte groot. Deze angst verdween echter met de serendipiteit van penicilline en de ontdekking van andere antibiotica. Soldaten in WO II hoefden niet meer te vrezen voor een infectie van hun verwondingen. Penicilline werd beschouwd als een wondermiddel en werd derhalve ook overvloedig gebruikt. Opgemerkt de vermindering van de effectiviteit penicilline haastten de artsen zich om nieuwe antibiotica te ontwikkelen. Deze ontwikkelingen in de 20\textsuperscript{ste} eeuw konden het ontstaan van antibiotica resistentie(AR) door irrationeel gebruik bijbenen. In 21\textsuperscript{ste} eeuw stagneert de ontwikkeling van nieuwe antibiotica echter. Met het huidig voorschrijfgedrag van huisartsen duurt het niet lang tot we weer in ``pre-penicilline periode'' bevinden. Het aantal sterftegevallen aan infectieziektes zal van hedendaagse 700.000 doden per jaar naar 10 miljoen doden per jaar in 2050 stijgen. Het floreren van elke natie komt in het geding.

Dus een bijscholing omtrent correcte antibiotica voorschrijving is noodzakelijk. In de aangeleverde factsheet wordt het AR problematiek en de bijscholing nader toegelicht. Het speerpunt van bijscholing is het vatten van de complementaire concepten ``disease'' en ``illness''. 

\closing{In anticipatie van een instemmend respons}
\encl{Factsheet over antibiotica resistentie}
\cc{Raad van Bestuur\\
Alle leden%
}
\end{letter}
\end{document}
